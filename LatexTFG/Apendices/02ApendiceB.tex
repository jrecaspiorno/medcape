%---------------------------------------------------------------------
%
%                          Ap�ndice 2
%
%---------------------------------------------------------------------
%
% 01AsiSeHizo.tex
% Copyright 2009 Marco Antonio Gomez-Martin, Pedro Pablo Gomez-Martin
%
% This file belongs to the TeXiS manual, a LaTeX template for writting
% Thesis and other documents. The complete last TeXiS package can
% be obtained from http://gaia.fdi.ucm.es/projects/texis/
%
% Although the TeXiS template itself is distributed under the 
% conditions of the LaTeX Project Public License
% (http://www.latex-project.org/lppl.txt), the manual content
% uses the CC-BY-SA license that stays that you are free:
%
%    - to share & to copy, distribute and transmit the work
%    - to remix and to adapt the work
%
% under the following conditions:
%
%    - Attribution: you must attribute the work in the manner
%      specified by the author or licensor (but not in any way that
%      suggests that they endorse you or your use of the work).
%    - Share Alike: if you alter, transform, or build upon this
%      work, you may distribute the resulting work only under the
%      same, similar or a compatible license.
%
% The complete license is available in
% http://creativecommons.org/licenses/by-sa/3.0/legalcode
%
%---------------------------------------------------------------------

\chapter{Recursos Android}
\label{recursos}

\begin{FraseCelebre}
\begin{Frase}
Donde hay educaci�n no hay distinci�n de clases
\end{Frase}
\begin{Fuente}
Confucio
\end{Fuente}
\end{FraseCelebre}

\begin{resumen}
Este ap�ndice recoge en detalle los fragmentos m�s importantes relacionados con el correcto funcionamiento de la aplicaci�n Android.
\end{resumen}

%-------------------------------------------------------------------
\section{Aplicaci�n Python}
%-------------------------------------------------------------------
\label{ap1:python}

A continuaci�n se muestra el c�digo de la aplicacaci�n Python utilizada para el env�o de datos a la aplicaci�n Android.

\lstset{language=Python}          % Set your language (you can change the language for each code-block optionally)

\begin{lstlisting}  % Start your code-block
import bluetooth
import time
import sys
import os
print "Starting programm"
# Open Bluetooth socket and listen for connections
serverSocket = bluetooth.BluetoothSocket(bluetooth.RFCOMM)
serverSocket.bind(("", bluetooth.PORT_ANY))
serverSocket.listen(1)  # Listen for any incoming connections
serverSocketChannel = serverSocket.getsockname()[1]
uuid = "34B1CF4D-1069-4AD6-89B6-E161D79BE4D8"
# Advertise our Bluetooth service so other devices can find it
bluetooth.advertise_service(serverSocket, 
		"BeagleBoneService",
        service_id = uuid,
        service_classes = [uuid, bluetooth.SERIAL_PORT_CLASS],
        profiles = [bluetooth.SERIAL_PORT_PROFILE])
pipeName = "pipe_medcape.data"
TAG_SIZE          = 2
ADC_CHANNELS      = 8
ADC_SAMP_CHANNEL  = 2
ADC_PACKET_SIZE   = int(TAG_SIZE + (ADC_CHANNELS * ADC_SAMP_CHANNEL)) #18 bytes per sample (2 status + 16 data). Each channel has 2 bytes
chunkSize = ADC_PACKET_SIZE
chunkSize = 18
while True:
	# Wait for an incoming connection
	print "Waiting for a new connection on channel", serverSocketChannel
	clientSocket, clientAddress = serverSocket.accept()
	print "Accepted connection from ", clientAddress
	time.sleep(1)
	# Send the data received from pipe to connected device
	try:
		pipeFile = open(pipeName, "rb")
		print pipeName, "opened"
		while True:
			try:
				while True:
					chunk = pipeFile.read(chunkSize)
					if len(chunk) <= 0:
						break
					else:
						clientSocket.send(chunk)
			except:
				break  # Device is disconnected
	except IOError:
		print "Pipe cannot be established"
	print "Device disconnected"
	clientSocket.close()
\end{lstlisting} 

%-------------------------------------------------------------------
\section{Recepci�n de datos}
%-------------------------------------------------------------------
\label{ap1:recep}

El fragmento de c�digo m�s representativo de la recepci�n de datos por parte de la aplicaci�n Android puede observarse a continuaci�n.

\lstset{language=Java}          % Set your language (you can change the language for each code-block optionally)

\begin{lstlisting}
void beginListen() {
		stopWorker = false;
		workerThread = new Thread(new Runnable() {
			@Override
			public void run() {
				while (!Thread.currentThread().isInterrupted() && !stopWorker) {
					if (BTConnection.receiveDataIsAvailable()) {
						recvBuff = BTConnection.receiveData();
						bytesRead = BTConnection.getBytesRead();
						if (send_transmission_online) {
							byte[] recvBuff_send_socket = new byte[bytesRead];
							for (int i = 0; i < bytesRead; i++) {
								if (i < recvBuff.length) {
									recvBuff_send_socket[i] = recvBuff[i];
								}
							}
							JSONObject data = new JSONObject();
							try {
								data.put("doctor_socket_id", doctor_socket_id);
								data.put("transmission", recvBuff_send_socket);
								ConfigSocketIO.socket.emit("emit transmission server", data);
							} catch (JSONException e) {
								System.out.println(e);
							}
						}
						for (int i = 0; i < bytesRead; i++) {
							if (i < recvBuff.length){
								ringBuffer.add(recvBuff[i]);
							}
						}
					}
					while (ringBuffer.size() > MIN_RING_BUFFER_LENGTH) {
						double point;
						//BTConnection.sendData(1);
						byte byte_get = ringBuffer.get();
						switch (byte_get) {
							case PACKET_TAG:
									//graphical representation fragment
								break;
							default:
								add_send_samples_online(byte_get);
								Log.w("", "Tag missed");
								break;
						}
					}
				}
			}
		});
		workerThread.start();
	}
\end{lstlisting}

%-------------------------------------------------------------------
\section{C�digo servidor}
%-------------------------------------------------------------------
\label{ap1:server}

A continuaci�n se muestran algunos de los fragmentos m�s representativos del c�digo que se ejecuta en el servidor.

\lstset{style=customjava}          % Set your language (you can change the language for each code-block optionally)

\begin{lstlisting}
socket.on('request transmission server', function(data){
		var patient_socket_id = users[data.patient_name];
		console.log('request transmission server');
		console.log("Doctor: ", socket.id);
		console.log("Patient: ", data.patient_name, " with id ", patient_socket_id);
		var data_to_client = {};
		data_to_client["doctor_socket_id"] = socket.id;
		data_to_client["patient_socket_id"] = patient_socket_id;
		socket.broadcast.to(patient_socket_id).emit(
		'request transmission client',{doctor_socket_id: socket.id});
	});

socket.on('emit transmission server', function(data){
		var doctor_socket_id = data.doctor_socket_id;
		console.log('emit transmission server');
		console.log("Patient: ", socket.id);
		console.log("Doctor: ", doctor_socket_id);
		console.log("transmission: ", data.transmission[0]);
		var users_keys = Object.keys(users);
		var user_values = users_keys.map(function(key) {
			return users[key];
		});
		var index_user =  user_values.indexOf(socket.id);
		var patient_name = users_keys[index_user];
		socket.broadcast.to(doctor_socket_id).emit(
		'receive transmission client',{transmission: data.transmission});
	});	

\end{lstlisting}




% Variable local para emacs, para  que encuentre el fichero maestro de
% compilaci�n y funcionen mejor algunas teclas r�pidas de AucTeX
%%%
%%% Local Variables:
%%% mode: latex
%%% TeX-master: "../ManualTeXiS.tex"
%%% End:
