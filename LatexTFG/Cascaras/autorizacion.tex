%---------------------------------------------------------------------
%
%                      resumen.tex
%
%---------------------------------------------------------------------
%
% Contiene el cap�tulo del resumen.
%
% Se crea como un cap�tulo sin numeraci�n.
%
%---------------------------------------------------------------------

\chapter{Autorizaci�n}
\cabeceraEspecial{Resumen}

\begin{FraseCelebre}
\begin{Frase}

\end{Frase}
\begin{Fuente}

\end{Fuente}
\end{FraseCelebre}

\begin{center}
\textbf{Autorizaci�n de difusi�n y utilizaci�n.}
\end{center}

\vspace*{\fill}
\begin{center}
\textbf{Autores:}
\newline
\newline
\newline
\newline
\newline
\textbf{Firmas:}
\newline
\newline
\newline
\newline
\newline
\textbf{Fecha:}
\newline
\end{center}




\endinput
% Variable local para emacs, para  que encuentre el fichero maestro de
% compilaci�n y funcionen mejor algunas teclas r�pidas de AucTeX
%%%
%%% Local Variables:
%%% mode: latex
%%% TeX-master: "../Tesis.tex"
%%% End:
