
%---------------------------------------------------------------------
%
%                          Cap�tulo 5
%
%---------------------------------------------------------------------

\chapter{Resultados}

\begin{FraseCelebre}
\begin{Frase}
S� el cambio que quieres ver en el mundo
\end{Frase}
\begin{Fuente}
Mahatma Gandhi
\end{Fuente}
\
\end{FraseCelebre}

\begin{resumen}
En este cap�tulo se exponen los resultados finales del proyecto, as� como se presenta una discusi�n cr�tica y razonada al respecto, ilustrado las conclusiones finales de los autores.
\end{resumen}

%-------------------------------------------------------------------
\section{Conclusiones}
%-------------------------------------------------------------------
\label{cap5:sec:conclusiones}

El trabajo aqu� presentado se enmarca dentro del campo de la medicina. Dentro de este amplio campo podemos encontrar todo tipo de soluciones y aplicaciones que tratan de facilitar la vida al usuario, en este caso concreto, al paciente.

Tras la realizaci�n del proyecto, se ha obtenido un producto final que podr�a ser considerado como un prototipo. Este prototipo podr�a ser adaptado y mejorado con el fin de adaptarse aun m�s a las necesidades tanto de pacientes como de doctores. Consideramos que el producto final podr�a ser probado en un contexto m�dico real con altas posibilidades de aplicarse exitosamente. De esta forma, podr�a el proyecto final ir incorpor�ndose poco a poco al mundo sanitario.

Entre los posibles usos reales que podr�a adquirir el resultado de este trabajo en el futuro, podemos destacar su uso en hospitales, en pacientes con movilidad reducida con ciertas incapacidades que impidan su comparecencia en el hospital, e incluso en personas mayores con alto riesgo de padecer enfermedades card�acas con el fin de prevenirlas y tratarlas de forma precoz.

En definitiva, gracias a la realizaci�n de este trabajo, nos ha sido posible adentrarnos m�nimamente en el campo de la medicina. Consideramos que todo lo relacionado con la medicina es un acontecimiento emocionante, ya que cualquier avance o mejora que se desarrolle, por m�nimo que sea, es capaz de mejorar potencialmente la calidad de vida de multitud de personas. Aunque este sea el primer proyecto importante que realizamos �ntimamente relacionado con este campo, no nos cabe duda de que no ser� el �ltimo.

%-------------------------------------------------------------------
\section{Conclusions}
%-------------------------------------------------------------------
\label{cap5:sec:conclusions}

Aqu� van las conclusiones en ingl�s.


%\medskip

%Y tambi�n ponemos el acr�nimo \ac{CVS} \footnote{M�s informaci�n sobre UART e I2C disponible en \href{https://geekytheory.com/puertos-y-buses-1-i2c-y-uart}{https://geekytheory.com/puertos-y-buses-1-i2c-y-uart}}para que no cruja.


%-------------------------------------------------------------------
%\section*{\NotasBibliograficas}
%-------------------------------------------------------------------
%\TocNotasBibliograficas

%Citamos algo para que aparezca en la bibliograf�a\ldots
%\citep{ldesc2e}

%\medskip

%Y tambi�n ponemos el acr�nimo \ac{CVS} para que no cruja.

%Ten en cuenta que si no quieres acr�nimos (o no quieres que te falle la compilaci�n en ``release'' mientras no tengas ninguno) basta con que no definas la constante \verb+\acronimosEnRelease+ (en \texttt{config.tex}).


%-------------------------------------------------------------------
%\section*{\ProximoCapitulo}
%-------------------------------------------------------------------
%\TocProximoCapitulo

%...

% Variable local para emacs, para  que encuentre el fichero maestro de
% compilaci�n y funcionen mejor algunas teclas r�pidas de AucTeX
%%%
%%% Local Variables:
%%% mode: latex
%%% TeX-master: "../Tesis.tex"
%%% End:
