%---------------------------------------------------------------------
%
%                          Cap�tulo 2
%
%---------------------------------------------------------------------

\chapter{Antecedentes}

\begin{FraseCelebre}
\begin{Frase}
Cada d�a sabemos m�s \\
y entendemos menos. 
\end{Frase}
\begin{Fuente}
Albert Einstein
\end{Fuente}
\end{FraseCelebre}

\begin{resumen}
\todo Aqui va el resumen de este cap�tulo una vez este terminado.
\end{resumen}


%-------------------------------------------------------------------
\section{Electrocardiograma}
%-------------------------------------------------------------------
\label{cap2:sec:electro}

Un electrocardiograma (m�s popularmente conocido como ECG) es un proceso por el cual se registran las actividades el�ctricas que emite el coraz�n durante un tiempo determinado.

El registro de esta actividad cardiaca es posible gracias a las diferencias de potencial existentes producidas por la contractilidad del coraz�n. El estudio de la informaci�n ilustrada por un ECG puede ser de gran utilidad a la hora de detectar multitud de enfermedades cardiovasculares, as� como prevenir problemas cardiacos de forma precoz.

La ventaja de un ECG frente a otros m�todos de medici�n para comprobar el estado del coraz�n, es que el ECG aporta mucha m�s informaci�n que los m�todos m�s habituales como pueden ser simplemente medir el pulso o utilizar un estetoscopio.

Entre la informaci�n extra que puede obtenerse recurriendo al ECG, cabe destacar desde la medici�n continua durante un tiempo prolongado (d�as incluso si se utiliza un ECG port�til) hasta la obtenci�n del movimiento de los m�sculos del coraz�n (los electrodos conectados al paciente registran esos movimientos en \textit{mV}), lo cual permite saber cuando entra la sangre, cuando sale, cuanto dura cada movimiento, etc. Sin esa informaci�n ser�a imposible detectar ciertos tipos de arritmias y/o otras alteraciones del coraz�n.

%-------------------------------------------------------------------
\section{Estructura de un ECG}
%-------------------------------------------------------------------
\label{cap2:sec:estructura}

La estructura habitual de un ECG [Fig \ref{fig:ondaecg}] habitualmente est� formada por un conjunto de ondas y complejos determinados:

\begin{itemize}  
\item Onda P 
\item Onda Q 
\item Complejo QRS
\item Onda T
\item Onda U
\end{itemize}

\figura{Vectorial/ecg}{width=.7\textwidth}{fig:ondaecg}%
{Estructura habitual de la se�al ECG durante un ciclo cardiaco.}

Habitualmente este tipo de se�ales se encuentran dentro de un rango determinado de amplitudes, que generalmente abarca desde los 0.5\textit{mV} hasta los 5\textit{mV}, existiendo adem�s una componente continua causada por el contacto existente entre los electrodos y la piel. 

Mayormente este tipo de se�ales suele ilustrarse en los libros de forma muy clara y reconocible, aunque no siempre es posible disponer de un entorno con las caracter�sticas propicias para eliminar toda existencia de ruido en la se�al. Generalmente el ruido que se recoge al analizar este tipo de se�ales es despreciado, aunque en ciertas ocasiones puede llegar a ser tan difuso, que nos impida reconocer hasta las pautas m�s caracter�sticas de una se�al ECG.

Puede presentarse ruido en la se�al simplemente debido a la luz tanto natural como artificial que incida indirectamente en los electrodos, as� como debido a la corriente que reciben los dispositivos que nos permiten llevar a cabo la medici�n de la se�al.

%-------------------------------------------------------------------
%\section*{\NotasBibliograficas}
%-------------------------------------------------------------------
%\TocNotasBibliograficas

%Citamos algo para que aparezca en la bibliograf�a\ldots
%\citep{ldesc2e}

%\medskip

%Y tambi�n ponemos el acr�nimo \ac{CVS} para que no cruja.

%Ten en cuenta que si no quieres acr�nimos (o no quieres que te falle la compilaci�n en ``release'' mientras no tengas ninguno) basta con que no definas la constante \verb+\acronimosEnRelease+ (en \texttt{config.tex}).


%-------------------------------------------------------------------
%\section*{\ProximoCapitulo}
%-------------------------------------------------------------------
%\TocProximoCapitulo

%...

% Variable local para emacs, para  que encuentre el fichero maestro de
% compilaci�n y funcionen mejor algunas teclas r�pidas de AucTeX
%%%
%%% Local Variables:
%%% mode: latex
%%% TeX-master: "../Tesis.tex"
%%% End:
