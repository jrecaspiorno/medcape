%---------------------------------------------------------------------
%
%                      resumen.tex
%
%---------------------------------------------------------------------
%
% Contiene el cap�tulo del resumen.
%
% Se crea como un cap�tulo sin numeraci�n.
%
%---------------------------------------------------------------------

\chapter{Resumen}
\cabeceraEspecial{Resumen}

\begin{FraseCelebre}
\begin{Frase}
Encu�ntrate y s� t� mismo; \\
recuerda que no hay nadie como t�
\end{Frase}
\begin{Fuente}
Dale Carnegie
\end{Fuente}
\end{FraseCelebre}

El prop�sito fundamental del proyecto es conseguir la lectura de muestras en tiempo real con unos plazos de captura de las mismas muy estrictos, con el fin de no perder ninguna muestra. El proyecto parte de una aplicaci�n en espacio de usuario que presentaba unas p�rdidas en la recepci�n de muestras de entre un 1 y un 2\% respecto al conjunto total de muestras enviadas originalmente. Se observ� que si adem�s de ejecutar esta aplicaci�n en espacio de usuario, se ejecutaba otra tarea en paralelo, el n�mero de muestras perdidas se disparaba. Era necesario realizar una descarga de la CPU.

Para conseguirlo, se ha utilizado la Unidad Programable en Tiempo-real de la BeagleBone Black, que es un hardware espec�ficamente dise�ado para este tipo de operaciones, es decir, para realizar tareas sencillas en tiempo real; principalmente se programa en ensamblador y facilita la interacci�n entre el c�digo ensamblador y una aplicaci�n en espacio de usuario.

Una vez implementada la soluci�n con �xito, se evita la p�rdida muestras en el proceso. Adem�s, se consigue descargar la CPU, de manera que es posible ejecutar otras tareas de forma paralela.

Finalmente, se consigue que la BeagleBone Black se comunique con una aplicaci�n Android mediante Bluetooth. Esta aplicaci�n facilita a un paciente la recogida de informaci�n de su electrocardiograma, para que posteriormente, o incluso en tiempo real, sea un m�dico el que trate de analizarla e interpretarla. Para que la comunicaci�n entre m�dico y paciente sea posible, es necesario que ambos dispongan de conexi�n a internet, ya que la interacci�n se realiza a trav�s de un servidor, el cual recibe la informaci�n de un dispositivo y la retransmite al otro. 

\break
\textbf{Palabras clave:} ECG, BeagleBone Black, PRU, ADS1198, Android



\endinput
% Variable local para emacs, para  que encuentre el fichero maestro de
% compilaci�n y funcionen mejor algunas teclas r�pidas de AucTeX
%%%
%%% Local Variables:
%%% mode: latex
%%% TeX-master: "../Tesis.tex"
%%% End:
