%---------------------------------------------------------------------
%
%                          Ap�ndice 1
%
%---------------------------------------------------------------------
%
% 01AsiSeHizo.tex
% Copyright 2009 Marco Antonio Gomez-Martin, Pedro Pablo Gomez-Martin
%
% This file belongs to the TeXiS manual, a LaTeX template for writting
% Thesis and other documents. The complete last TeXiS package can
% be obtained from http://gaia.fdi.ucm.es/projects/texis/
%
% Although the TeXiS template itself is distributed under the 
% conditions of the LaTeX Project Public License
% (http://www.latex-project.org/lppl.txt), the manual content
% uses the CC-BY-SA license that stays that you are free:
%
%    - to share & to copy, distribute and transmit the work
%    - to remix and to adapt the work
%
% under the following conditions:
%
%    - Attribution: you must attribute the work in the manner
%      specified by the author or licensor (but not in any way that
%      suggests that they endorse you or your use of the work).
%    - Share Alike: if you alter, transform, or build upon this
%      work, you may distribute the resulting work only under the
%      same, similar or a compatible license.
%
% The complete license is available in
% http://creativecommons.org/licenses/by-sa/3.0/legalcode
%
%---------------------------------------------------------------------

\chapter{C�digo}
\label{codigo}

\begin{FraseCelebre}
\begin{Frase}
Pones tu pie en el camino y si no cuidas tus pasos, nunca sabes a donde te pueden llevar.
\end{Frase}
\begin{Fuente}
John Ronald Reuel Tolkien, El Se�or de los Anillos
\end{Fuente}
\end{FraseCelebre}

\begin{resumen}
Este ap�ndice cuenta algunos de los aspectos t�cnicos m�s importantes del proyecto.
\end{resumen}

%-------------------------------------------------------------------
\section{Edici�n}
%-------------------------------------------------------------------
\label{ap1:edicion}

Eso era en realidad un s�ntoma indicativo de que en nuestro trabajo
cotidiano utilizamos emacs para editar ficheros \LaTeX. Es cierto que
inicialmente utilizamos otros editores creados expresamente para la
edici�n de ficheros en \LaTeX, pero descubrimos emacs y ha llegado
para quedarse.





% Variable local para emacs, para  que encuentre el fichero maestro de
% compilaci�n y funcionen mejor algunas teclas r�pidas de AucTeX
%%%
%%% Local Variables:
%%% mode: latex
%%% TeX-master: "../ManualTeXiS.tex"
%%% End:
