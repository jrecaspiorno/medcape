%---------------------------------------------------------------------
%
%                          Cap�tulo 1
%
%---------------------------------------------------------------------

\chapter{Introducci�n}

\begin{FraseCelebre}
\begin{Frase}
Un comienzo no desaparece nunca, \\
ni siquiera con un final.
\end{Frase}
\begin{Fuente}
Harry Mulisch
\end{Fuente}
\end{FraseCelebre}

%\begin{resumen}

%\end{resumen}

%-------------------------------------------------------------------
\section{Motivaci�n}
%-------------------------------------------------------------------
\label{cap1:sec:motivacion}

La sociedad actual en la que nos encontramos est� completamente informatizada podr�amos decir. Es posible encontrar software en todo tipo de lugares en los que jam�s hubiesemos pensado hace d�cadas que hubiese sido posible, como por ejemplo, neveras en las que una vez acabado un cierto tipo de refrigerio, es el propio electrodom�stico el encargado de comprarlo por nosotros, relojes con los que podamos conectarnos a internet y comunicarnos, eliminando la necesidad de llevar un tel�fono encima, e incluso zapatillas que se encargan de pedir nuestra comida favorita a domicilio con tan solo pulsar un peque�o bot�n.

Es imposible enumerar la cantidad de aplicaciones que el software podr�a tener, al menos en la presente memoria, y todo esto sin olvidar los futuros usos que adquirir�. Podr�amos decir que la industr�a del Software est� en pleno auge, es m�s, lleva en pleno auge desde hace d�cadas, y no parece que vaya a decaer. Con tantos posibles sectores en los que especializarse, parece complicado elegir uno en el que sumergirse. 

Sin embargo, para los autores de la presente memoria siempre tuvo cierto atractivo el desarrollo de software para dispositivos empotrados que utilizasen software libre. Fue entonces cuando se nos dio la posibilidad de trabajar en el presente proyecto, utilizando este tipo de dispotivos y adem�s �ntimamente enfocado y relacionado con el cuidado de la salud. La mezcla de ambos componentes nos fascin� a primera vista.

El desarrollo de software para dispositivos empotrados se encuentra en pleno crecimiento desde hace ya varios a�os, y cada vez son m�s populares las comunidades que dan soporte a los desarrolladores de los mismos, facilitando as� el proceso de creaci�n del software. Asimismo tambi�n se encuentra en uno de sus mejores momentos todo tipo de \textit{gadget} capaz de medir o monitorizar ciertas se�ales, como pueden ser los actuales relojes inteligentes, o las pulseras de actividad tan frecuentemente vistas. Estos dispositivos son capaces de monitorizar multitud de se�ales procedentes del cuerpo humano, y todo esto en un reducido espacio f�sico f�cilmente portable.

La monitorizaci�n de ciertas se�ales biom�dicas puede ser un factor fundamental a la hora de detectar problemas de salud de forma precoz. La presente memoria trata de ilustrar el proceso gracias al cual es posible monitorizar este tipo de se�ales utilizando un hardware de bajo coste y portable, y un software libre, adecuado y preciso, cuya uni�n pueda facilitar la pr�ctica de este tipo de procesos en todos los contextos en los que pudiera ser necesario.



%-------------------------------------------------------------------
%\section{Estructura de cap�tulos}
%-------------------------------------------------------------------
%\label{cap1:sec:estructura}

%-------------------------------------------------------------------
%\section*{\NotasBibliograficas}
%-------------------------------------------------------------------
%\TocNotasBibliograficas

%Citamos algo para que aparezca en la bibliograf�a\ldots
%\citep{ldesc2e}

%\medskip

%Y tambi�n ponemos el acr�nimo \ac{CVS} para que no cruja.

%Ten en cuenta que si no quieres acr�nimos (o no quieres que te falle la compilaci�n en ``release'' mientras no tengas ninguno) basta con que no definas la constante \verb+\acronimosEnRelease+ (en \texttt{config.tex}).


%-------------------------------------------------------------------
%\section*{\ProximoCapitulo}
%-------------------------------------------------------------------
%\TocProximoCapitulo


% Variable local para emacs, para  que encuentre el fichero maestro de
% compilaci�n y funcionen mejor algunas teclas r�pidas de AucTeX
%%%
%%% Local Variables:
%%% mode: latex
%%% TeX-master: "../Tesis.tex"
%%% End:
