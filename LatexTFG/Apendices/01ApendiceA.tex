%---------------------------------------------------------------------
%
%                          Ap�ndice 1
%
%---------------------------------------------------------------------
%
% 01AsiSeHizo.tex
% Copyright 2009 Marco Antonio Gomez-Martin, Pedro Pablo Gomez-Martin
%
% This file belongs to the TeXiS manual, a LaTeX template for writting
% Thesis and other documents. The complete last TeXiS package can
% be obtained from http://gaia.fdi.ucm.es/projects/texis/
%
% Although the TeXiS template itself is distributed under the 
% conditions of the LaTeX Project Public License
% (http://www.latex-project.org/lppl.txt), the manual content
% uses the CC-BY-SA license that stays that you are free:
%
%    - to share & to copy, distribute and transmit the work
%    - to remix and to adapt the work
%
% under the following conditions:
%
%    - Attribution: you must attribute the work in the manner
%      specified by the author or licensor (but not in any way that
%      suggests that they endorse you or your use of the work).
%    - Share Alike: if you alter, transform, or build upon this
%      work, you may distribute the resulting work only under the
%      same, similar or a compatible license.
%
% The complete license is available in
% http://creativecommons.org/licenses/by-sa/3.0/legalcode
%
%---------------------------------------------------------------------

\chapter{Recursos software}
\label{recursos}

\begin{FraseCelebre}
\begin{Frase}
Pones tu pie en el camino y si no cuidas tus pasos, nunca sabes a donde te pueden llevar.
\end{Frase}
\begin{Fuente}
John Ronald Reuel Tolkien, El Se�or de los Anillos
\end{Fuente}
\end{FraseCelebre}

\begin{resumen}
Este ap�ndice recoge en detalle los fragmentos m�s importantes para garantizar el correcto funcionamiento del proyecto.
\end{resumen}

%-------------------------------------------------------------------
\section{Device Tree Overlay}
%-------------------------------------------------------------------
\label{ap1:dto}

El resultado final del DTO que se obtuvo finalmente y que es utilizado actualmente en la BBB se ilustra en el siguiente c�digo. El c�digo final fue comentado de modo que pudiese ser interpretado y comprendido de forma sencilla.

\lstset{language=Pascal}          % Set your language (you can change the language for each code-block optionally)

\begin{lstlisting}  % Start your code-block

/dts-v1/;
/plugin/;
/ {
   compatible = "ti,beaglebone", "ti,beaglebone-black";
   part-number = "EBB-PRU-ADC2";
   version = "00A0";
   /* This overlay uses the following resources */
   exclusive-use =
          "P9.24", "P9.25", "P9.27", "P9.28", "P9.29",
           "P9.30", "P9.31", "P8.46", "pru0", "pru1";
   fragment@0 {
      target = <&am33xx_pinmux>;
      __overlay__ {
         pru_pru_pins: pinmux_pru_pru_pins {  
				0x184 0x2e  // DRDY P9_24 pr1_pru0_pru_r31_16,
				 MODE6 | INPUT  | DIS  00101110=0x2e
				0x1ac 0x0d  // START   P9_25 pr1_pru0_pru_r30_7,
				 MODE5 | OUTPUT | DIS  00001101=0x0d
				0x190 0x0d  // RESET   P9_31 pr1_pru0_pru_r30_0,
				 MODE5 | OUTPUT | DIS  00001101=0x0d
               0x1a4 0x0d  // CS   P9_27 pr1_pru0_pru_r30_5,
                MODE5 | OUTPUT | DIS  00001101=0x0d
               0x19c 0x2e  // MISO P9_28 pr1_pru0_pru_r31_3,
                MODE6 | INPUT  | DIS  00101110=0x2e
               0x194 0x0d  // MOSI P9_29 pr1_pru0_pru_r30_1,
                MODE5 | OUTPUT | DIS  00001101=0x0d
               0x198 0x0d  // CLK  P9_30 pr1_pru0_pru_r30_2,
                MODE5 | OUTPUT | DIS  00001101=0x0d
	       // This is for PRU1, the sample clock -- debug 
				0x0a4 0x0d  // SAMP P8_46 pr1_pru1_pru_r30_1,
				 MODE5 | OUTPUT | DIS  00001101=0x0d
            >;
         };
      };
   };
   fragment@1 {         // Enable the PRUSS
      target = <&pruss>;
      __overlay__ {
         status = "okay";
         pinctrl-names = "default";
         pinctrl-0 = <&pru_pru_pins>;
      };
   };
};
\end{lstlisting}

%-------------------------------------------------------------------
\section{Inicializaci�n del ADS1198}
%-------------------------------------------------------------------
\label{ap1:ini}

En el siguiente fragmento de c�digo ilustra el proceso de inicializaci�n del ADS del proyecto original. Es un c�digo relativamente sencillo y completamente funcional. Los pasos que se siguen en el c�digo original, desde resetear el ciclo, hasta la lectura de datos en modo continuo, son replicados en c�digo ensamblador. El objetivo final ha sido conseguir la misma correcta funcionalidad de la que gozaba el c�digo original.

\lstset{style=customc}          % Set your language (you can change the language for each code-block optionally)

\begin{lstlisting}  % Start your code-block
	printf("Init GPIOs\n");
    if ( ads_init_gpios() == -1 ) {
        printf("can't init GPIOs\n");
        return -1;
    }
    printf("Conf SPI\n");
    init_spi(spi_hz>>5);
    sleep(1);
    printf("Reset cycle\n");
    if ( ads_reset() == -1 ) {
        printf("can't reset ADS\n");
        return -1;
    }
    printf("Stop Read Data Continuously mode\n");
    if ( ads_sdatac() == -1 ) {
        printf("can't stop read data continuously\n");
        return -1;
    }
    sleep(1);
    printf("POR registers\n");
    printf("Set sample rate %d\n", tmp_srate);
    if ( ads_set_rate(sample_rate) == -1 ) {
        printf("can't set sample rate!!\n");
        return -1;
    }
    sleep(1);
    printf("Enable internal reference\n");
    if ( ads_enable_intref() == -1 ) {
        printf("Error enabling internal reference!!\n");
        return -1;
    }
    sleep(2);
    printf("Checl ID\n");
    if ( ads_check_id() == -1 ) {
        printf("can't read IDReg\n");
        return -1;
    }
    if(test_signal){
        printf("Set test signals\n");
        if( ads_set_test() == -1 ){
            printf("can't read IDReg\n");
            return -1;
        }
    }
    ads_print_registers();
    printf("START: start continuous reading\n");
    if ( ads_start() == -1 ) {
        printf("can't start conversion\n");
        return -1;
    }
    printf("RDATAC: read data continuously\n");
    if ( ads_command(RDATAC) == -1 ) {
        printf("can't read data continuously\n");
        return -1;
    }
    close_spi();
    init_spi(spi_hz);
\end{lstlisting}

A continuaci�n se muestra el c�digo desarrollado en ensamblador, totalmente funcional, y utilizado en el presente proyecto, que imita a la perfecci�n el funcionamiento del c�digo original de inicializaci�n del ADS en C.

\lstset{style=customasm}          % Set your language (you can change the language for each code-block optionally)

\begin{lstlisting}  % Start your code-block
//----------------------SDATAC----------------------
SEND_SDATAC:
	LBBO	r2, r1, 0, 4	 // Load sdatac command=0x11
	//We only want 8 bits, but 1 register=32 bits 
	//so we do lsl(logical shift left) to move our LSB to MSB
	SUB r13, r14, 8
	CALL TRUNCATE
	CLR	r30.t5		 // set the CS line low (active low)
	MOV	r4, 8		 // going to write/read 8 bits (1 byte)
	CALL	SPICLK_LOOP           // repeat call the SPICLK 
	SET	r30.t5		 // pull the CS line high (end of sample)
//------------------SLEEP 2 SECONDS---------------------------
MOV	r12, DELAYCOUNT	
CALL DELAY_FUNCTION
\end{lstlisting}

\lstset{style=customasm}          % Set your language (you can change the language for each code-block optionally)

\begin{lstlisting}  % Start your code-block

//---------------------SET_ADS_SAMPLE_RATE	
SET_ADS_SAMPLE_RATE:
	//LBBO	r2, r1, 8, 12	 // Load sample rate speed
	MOV r2, 0x41
	//We only want 8 bits, but 1 register=32 bits
	//so we do lsl(logical shift left) to move our LSB to MSB
	SUB r13, r14, 8
	CALL TRUNCATE
	CLR	r30.t5		 // set the CS line low (active low)
	MOV	r4, 8		 // going to write/read 8 bits (1 byte)
	CALL	SPICLK_LOOP           // repeat call the SPICLK 
	MOV	r12, 1000 //Numero aleatorio
	CALL DELAY_FUNCTION
	MOV r2, 0x00000000
	MOV	r4, 8		 // going to write/read 8 bits (1 byte)
	CALL	SPICLK_LOOP           // repeat call the SPICLK 
	MOV	r12, 1000	 //Numero aleatorio
	CALL DELAY_FUNCTION	
	MOV r2, 0x06
	SUB r13, r14, 8
	CALL TRUNCATE
	MOV	r4, 8		 // going to write/read 8 bits (1 byte)
	CALL	SPICLK_LOOP           // repeat call the SPICLK 
	MOV	r12, 1000	//Numero aleatorio
	CALL DELAY_FUNCTION	
	SET	r30.t5		 // pull the CS line high (end of sample)
	SET r30.t1 //MOSI is active in original hardware SPI
//------------------SLEEP 2 SECONDS----
MOV	r12, DELAYCOUNT
CALL DELAY_FUNCTION

\end{lstlisting}

\lstset{style=customasm}          % Set your language (you can change the language for each code-block optionally)

\begin{lstlisting}  % Start your code-block

//---------------------SET_INTERNAL_REFERENCE_ADS
SET_INTERNAL_REFERENCE_ADS:
	//LBBO	r2, r1, 8, 12	 // Load sample rate speed
	MOV r2, 0x43
	SUB r13, r14, 8
	CALL TRUNCATE
	CLR	r30.t5		 // set the CS line low (active low)
	MOV	r4, 8		 // going to write/read 8 bits (1 byte)
	CALL	SPICLK_LOOP           // repeat call the SPICLK
	MOV	r12, 1000	//Numero aleatorio
	CALL DELAY_FUNCTION		
	MOV r2, 0x00000000
	MOV	r4, 8		 // going to write/read 8 bits (1 byte)
	CALL	SPICLK_LOOP           // repeat call the SPICLK 
	MOV	r12, 1000	//Numero aleatorio
	CALL DELAY_FUNCTION		
	MOV r2, 0xC0
	SUB r13, r14, 8
	CALL TRUNCATE
	MOV	r4, 8		 // going to write/read 8 bits (1 byte)
	CALL	SPICLK_LOOP           // repeat call the SPICLK 
	MOV	r12, 1000	//Numero aleatorio
	CALL DELAY_FUNCTION
	SET	r30.t5		 // pull the CS line high (end of sample)
	SET r30.t1 //MOSI is active in original hardware SPI
//------------------SLEEP 2 SECONDS-------
MOV	r12, DELAYCOUNT	
CALL DELAY_FUNCTION

\end{lstlisting}

\lstset{style=customasm}          % Set your language (you can change the language for each code-block optionally)

\begin{lstlisting}  % Start your code-block
//---------------------RDATAC----------------------
SEND_RDATAC:
	LBBO	r2, r1, 4, 8	 // Load rdatac command
	SUB r13, r14, 8
	CALL TRUNCATE
	CLR	r30.t5		 // set the CS line low (active low)
	MOV	r4, 8		 // going to write/read 8 bits (1 byte)
	CALL	SPICLK_LOOP           // repeat call the SPICLK 
	SET	r30.t5		 // pull the CS line high (end of sample)

\end{lstlisting}

%-------------------------------------------------------------------
\section{Comunicaci�n SPI}
%-------------------------------------------------------------------
\label{ap1:com}

El proceso de comunicaci�n SPI desarrollado en c�digo ensamblador, se ilustra en los siguientes framentos de c�digo. Cabe destacar que cada registro en ensamblador tiene una capacidad de 4 bytes. Debido al tama�o de cada muestra que se recoge, es necesario almacenar esta informaci�n en 7 registros. 

En el siguiente c�digo se muestra la funci�n SPICLK\_LOOP escrita en c�digo ensamblador, en la cual se refleja el proceso de iteraci�n sobre cada uno de los bits concretos, as� como el uso de todos los registros necesarios para el almacenamiento de la informaci�n.

\lstset{style=customasm}          % Set your language (you can change the language for each code-block optionally)

\begin{lstlisting}  % Start your code-block

GET_SAMPLE: //Receive 1 sample (18 bytes)
	CLR	r30.t5		 // set the CS line low (active low)
	MOV r15, 21 //Number of bytes/sample 
	GET_BYTE:
		MOV	r4, 8		 // going to write/read 1 byte (8 bits)
		MOV r2, 0x00000000 //What we want to write (i.e 0)
		CALL	SPICLK_LOOP // repeat call the SPICLKread		
		MOV	r12, 800 //Numero aleatorio
		CALL DELAY_FUNCTION
		SET	r30.t1 //MOSI: 1 (before start taking the sample)
		MOV	r12, 800 //Numero aleatorio
		CALL DELAY_FUNCTION
		CLR	r30.t1 //MOSI: 0 (before start taking the sample)
		SUB	r15, r15, 1     // decrement loop counter
	QBNE	GET_BYTE, r15, 0  // repeat loop unless zero
	SET	r30.t5		 // pull the CS line high (end of sample)
	SET	r30.t1 //MOSI: 1 (before start taking the sample)
	LSR	r3, r3, 1        // SPICLK shifts left 
	LSR	r17, r17, 1        // SPICLK shifts left 
	LSR	r18, r18, 1        // SPICLK shifts left
	LSR	r19, r19, 1        // SPICLK shifts left 
	LSR	r20, r20, 1        // SPICLK shifts left 
	LSR	r23, r23, 1        // SPICLK shifts left 
	LSR	r27, r27, 1        // SPICLK shifts left 


\end{lstlisting}

\lstset{style=customasm}          % Set your language (you can change the language for each code-block optionally)

\begin{lstlisting}  % Start your code-block

SPICLK_LOOP: //LOOP through the X bits (read and write)
	SPI_CLK_BIT:
			MOV	r0, r11	 // time for clock low 
		CLKLOW:	
			SUB	r0, r0, 1	 // decrement the counter by 1
		QBNE	CLKLOW, r0, 0	 // check if count low	
			// Take bit 31 in the next loop bit 31
			//So this way we take every bit we want
			QBBC	DATALOW, r2.t31  //QBC 
			SET	r30.t1 //The bit that we want is 1
			QBA	DATACONTD //QBA = Quick branch always
		DATALOW:
			CLR	r30.t1 //The bit that we want is 0
		DATACONTD:
			SET	r30.t2		 // set the clock high
			MOV	r0, r11	 // time for clock high
		CLKHIGH:
			SUB	r0, r0, 1	 // decrement the counter 
		QBNE	CLKHIGH, r0, 0	 // check the count
			LSL	r2, r2, 1 //Permorm logical shif left 
						 // clock goes low now read 
			CLR	r30.t2		 // set the clock low
			//----Store 1 received bit in a register 
			QBBC	DATAINLOW, r31.t3 //Take the MISO bit
				OR	r21, r21, 0x000001 //bit received=1
			DATAINLOW:	 //bit received=0
				LSL	r21, r21, 1 
		QBNE END_TAKING_MISO, r26, 1 //Flag 
			QBLE STORE_1ST_REGISTER, r22, 23
				QBA BREAK_IF //r22 > 23
			STORE_1ST_REGISTER: 
				QBNE STORE_2ND_REGISTER, r22, 23 
				MOV r3, r21 //If we wan to check bits
				MOV r21, 0x000000
			QBLE STORE_2ND_REGISTER, r22, (23*2)+1
				QBA BREAK_IF //r22 > 23
			STORE_2ND_REGISTER: //Entra aqui si r22 >= 23             
				QBNE STORE_3RD_REGISTER, r22, (23*2)+1
				MOV r17, r21 //If we wan to check bits
				MOV r21, 0x000000
			QBLE STORE_3RD_REGISTER, r22, (23*3)+2
				QBA BREAK_IF //r22 > 23
			STORE_3RD_REGISTER: //Entra aqui si r22 >= 23            
				QBNE STORE_4TH_REGISTER, r22, (23*3)+2 
				MOV r18, r21 //If we wan to check bits
				MOV r21, 0x000000
			QBLE STORE_4TH_REGISTER, r22, (23*4)+3
				QBA BREAK_IF //r22 > 23
			STORE_4TH_REGISTER: //Entra aqui si r22 >= 23            
				QBNE STORE_5TH_REGISTER, r22, (23*4)+3 
				MOV r19, r21 //If we wan to check how bits
				MOV r21, 0x000000
			QBLE STORE_5TH_REGISTER, r22, (23*5)+4
				QBA BREAK_IF //r22 > 23
			STORE_5TH_REGISTER: //Entra aqui si r22 >= 23             
				QBNE STORE_6TH_REGISTER, r22, (23*5)+4 
				MOV r20, r21 //If we wan to check bits
				MOV r21, 0x000000
			QBLE STORE_6TH_REGISTER, r22, (23*6)+5
				QBA BREAK_IF //r22 > 23	
			STORE_6TH_REGISTER: //Entra aqui si r22 >= 23             
				QBNE STORE_7TH_REGISTER, r22, (23*6)+5 
				MOV r23, r21 //If we wan to check bits
				MOV r21, 0x000000
			QBLE STORE_7TH_REGISTER, r22, (23*7)+6
				QBA BREAK_IF //r22 > 23
			STORE_7TH_REGISTER: //Entra aqui si r22 >= 23             
				QBNE BREAK_IF, r22, (23*7)+6 //23 < r22
				MOV r27, r21 //If we wan to check bits
				MOV r21, 0x000000
			BREAK_IF:
				ADD r22, r22, 1	//Entra aqui si r22 < 23
		END_TAKING_MISO:
		SUB	r4, r4, 1   // count down through the bits	
		QBNE	SPICLK_LOOP, r4, 0		
	RET	


\end{lstlisting}

%-------------------------------------------------------------------
\section{Guardado de muestras en memoria}
%-------------------------------------------------------------------
\label{ap1:guardado}

A continuaci�n se muestra el proceso de guardado de muestras desarrollado en c�digo ensamblador.

\lstset{style=customasm}          % Set your language (you can change the language for each code-block optionally)

\begin{lstlisting}  % Start your code-block

STORE_DATA:         // store the sample value in memory
	SUB	r9, r9, 21	 // reducing the number of bytes 
	MOV	r22, 0 //r3
	MOV	r26, 1	//r20
	MOV r5, r3
	CALL REVERSE_ENDIANNESS
	//-----Store in the whole sample 
	SBBO	r1, r8, 0, 2	//Last register only needs 2 bytes
	ADD		r8, r8, 2	// shifting RAM addres by 2 bytes 
	MOV	r12, 50 //Numero aleatorio
	CALL DELAY_FUNCTION
	MOV r5, r17
	CALL REVERSE_ENDIANNESS
	SBBO	r1, r8, 0, 3	
	ADD		r8, r8, 3	
	MOV	r12, 50 //Numero aleatorio
	CALL DELAY_FUNCTION
	MOV r5, r18
	CALL REVERSE_ENDIANNESS
	SBBO	r1, r8, 0, 3	 
	ADD		r8, r8, 3	 
	MOV	r12, 50 //Numero aleatorio
	CALL DELAY_FUNCTION
	MOV r5, r19
	CALL REVERSE_ENDIANNESS
	SBBO	r1, r8, 0, 3	 
	ADD		r8, r8, 3	
	MOV	r12, 50 //Numero aleatorio
	CALL DELAY_FUNCTION
	MOV r5, r20
	CALL REVERSE_ENDIANNESS
	SBBO	r1, r8, 0, 3	 
	ADD		r8, r8, 3	 // shifting RAM addres
	MOV	r12, 50 //Numero aleatorio
	CALL DELAY_FUNCTION
	MOV r5, r23
	CALL REVERSE_ENDIANNESS
	SBBO	r1, r8, 0, 3	 // store the value r3 
	ADD		r8, r8, 3	 // shifting RAM addres 
	MOV	r12, 50 //Numero aleatorio(
	CALL DELAY_FUNCTION
	MOV r5, r27
	CALL REVERSE_ENDIANNESS
	SBBO	r1, r8, 0, 1	 // store the value r3 
	ADD		r8, r8, 1	 // shifting RAM addres 
	MOV	r12, 50 //Numero aleatorio
	CALL DELAY_FUNCTION
	ADD r6, r6, 1
	QBNE CONTINUE_THIS_LOOP_RAM, r24, SIZE_CHUNK_RAM-1
		// generate an interrupt to notificate a new chunk 
		MOV r1, 0xFF
		SBBO	r1, r8, 0, 1	 // store the value r3 
		ADD		r8, r8, 1	 // shifting RAM addres by 4 bytes
		MOV r8, r25 //Reset ram direction 
		MOV r24, 0	
		MOV R31.b0, PRU0_R31_VEC_VALID | PRU_EVTOUT_1 
		//WAIT_FOR_ARM_TO_READ:
		WBS r31, #30
		QBA END_PROCESS_RAM_BUFFER
	CONTINUE_THIS_LOOP_RAM:
	ADD r24, r24, 1
	END_PROCESS_RAM_BUFFER:
		// clear registers to receive the response
	MOV	r3,  0x000000
	MOV	r17, 0x000000	 
	MOV	r18, 0x000000	
	MOV	r19, 0x000000	 
	MOV	r20, 0x000000	
	MOV	r23, 0x000000	
	MOV	r27, 0x000000	
	POLLING_DATA_READY_LOW: 
	QBBC	POLLING_DATA_READY_LOW, r31.t16 
	QBA	POLLING_DATA_READY_HIGH
END:

\end{lstlisting}

%-------------------------------------------------------------------
\section{Lectura de muestras en memoria}
%-------------------------------------------------------------------
\label{ap1:lectura}

A continuaci�n se muestra el proceso de lectura de muestras desde el programa de usuario desarrollado en C.

\lstset{style=customc}          % Set your language (you can change the language for each code-block optionally)

\begin{lstlisting}  % Start your code-block

int mem2file_main(int number_chunk, int samples_taken) {
    target = addr;
	int i = 0;
	stat(filename, &st);
	int size_file = st.st_size;
	if (size_file >= (size_packet*samples_per_chunk)*50000) {
		if ( (fd_output = freopen(filename, "w+b", fd_output))
		 == NULL ) {
			perror("Error al abrir el fichero de datos");
		}
	}
	while (((uint8_t *)pru1_data0)
	[(size_packet*samples_per_chunk)]
	 != 0xFF) {
		printf("Waiting for mem shared to refresh \n");
		fflush(stdout);
	   printf("Mem flag incorrect [%d] is: %x \n", 
	   (size_packet*samples_per_chunk),
	    ((uint8_t *)pru1_data0)
	    [(size_packet*samples_per_chunk)]);  
		fflush(stdout);
		printf("\n");
	}
   printf("Mem flag correct [%d] is: %x \n", 
   (size_packet*samples_per_chunk),
    ((uint8_t *)pru1_data0)[(size_packet*samples_per_chunk)]);  
	fwrite(pru1_data0, size_packet, 
	samples_per_chunk, fd_output);
	((uint8_t *)pru1_data0)[(size_packet*samples_per_chunk)]
	 = 0x00;
	// Try to open the Python pipe
	if (pipePython.fid < 1) {
		pipePython.fid = 
		open(pipePython.name, O_WRONLY | O_NONBLOCK);
		//printf("Pipe open\n");
	}
	// Send data if the Python pipe is open
	if (pipePython.fid > 0) {
		int writeSuccess = write(pipePython.fid, pru1_data0,
		 size_packet*samples_per_chunk);
		if (writeSuccess < 0) {
			close(pipePython.fid);
			pipePython.fid = -1;
		} else if (writeSuccess > PIPE_BUF) {
			printf("Warning!! Write in FIFO is not atomic\n");
		}
	}
    return 0;
}
\end{lstlisting}





% Variable local para emacs, para  que encuentre el fichero maestro de
% compilaci�n y funcionen mejor algunas teclas r�pidas de AucTeX
%%%
%%% Local Variables:
%%% mode: latex
%%% TeX-master: "../ManualTeXiS.tex"
%%% End:
